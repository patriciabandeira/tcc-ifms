% RESUMO--------------------------------------------------------------------------------

\begin{resumo}[RESUMO]
\begin{SingleSpacing}

% Não altere esta seção do texto--------------------------------------------------------
\imprimirautorcitacao. \imprimirtitulo. \imprimirdata. \pageref {LastPage} f. \imprimirprojeto\ – \imprimirprograma, \imprimirinstituicao. \imprimirlocal, \imprimirdata.\\
%---------------------------------------------------------------------------------------

Tendo em vista que há poucas informações disponíveis sobre as diversas espécies de plantas que predominam na região pantaneira, foi pensado em um meio de dispor tais informações a fim de auxiliar em pesquisas futuras e ofertar um conhecimento a mais para a população residente na região.
A disponibilização das informações ocorrerá por meio de uma aplicação mobile em forma de catálogo, fazendo o uso de uma nova tecnologia denominada React Native, pois ela permite a criação de aplicativos híbridos, ou seja, para android e IOS.
Foi escolhido o desenvolvimento de uma aplicação devido a alta utilização de aparelhos celulares nos dias atuais. A mesma  irá conter informações gerais como o nome científico e popular, ocorrência, distribuição geográfica, grau de ameaça,época de floração.
Espera-se, que com o desenvolvimento da aplicação, o  catálogo mobile possa contribuir com a população como uma forma de adquirir mais conhecimentos sobre a flora pantaneira e também auxiliar em pesquisas no ambiente acadêmico.\\


\textbf{Palavras-chave}: Palavra. Segunda Palavra. Outra palavra.

\end{SingleSpacing}
\end{resumo}

% OBSERVAÇÕES---------------------------------------------------------------------------
% Altere o texto inserindo o Resumo do seu trabalho.
% Escolha de 3 a 5 palavras ou termos que descrevam bem o seu trabalho 

